\noindent For example, for

\begin{equation}
	\mathbf{A}
	=
	\left(
	\begin{array}{cc}
		a_{11} & a_{12} \\
		a_{21} & a_{22} 
	\end{array}
	\right)
\end{equation}

\noindent the vectorization is given by

\begin{equation}
	\mathrm{vec}
	\left(
	\mathbf{A}
	\right)
	=
	\left(
	\begin{array}{c}
		a_{11} \\
		a_{21} \\ 
		a_{12} \\
		a_{22} 
	\end{array}
	\right),
\end{equation}

\noindent the half-vectorization is given by

\begin{equation}
	\mathrm{vech}
	\left(
	\mathbf{A}
	\right)
	=
	\left(
	\begin{array}{c}
		a_{11} \\
		a_{21} \\ 
		a_{22} 
	\end{array}
	\right),
\end{equation}

\noindent and the strict half-vectorization is given by

\begin{equation}
	\mathrm{vechs}
	\left(
	\mathbf{A}
	\right)
	=
	\left(
	\begin{array}{c}
		a_{21} 
	\end{array}
	\right).
\end{equation}